\section{Einleitung}

\subsection{�bersicht}
Kapitel 1 gibt einen kurzen �berblick �ber das Projekt. Kapitel 2 befasst sich mit dem IST-Zustand und den gew�nschten Funktionalit�ten f�r die Zukfunft aus Benutzersicht. In Kapitel 3 sind die Anforderungen aus Kapitel 2 st�rker strukturiert und aus einer mehr technischen Sicht beschrieben.

\subsection{Zweck}
Dieses Dokument befasst sich mit der Internetseite Incave.de und einer anstehenden Umsetzung aller in diesem Dokument genannten Funktionalit�ten in eine entsprechende Webanwendung.

\subsection{Ziel} 
Momentan sind die einzelnen Funktionalit�ten, welche Incave.de bereitstellt �ber verschiedene, teils von unterschiedlichen Anbietern hergestellten, PHP-Scripte realisiert. Dies f�hrt zu einem hohen Wartungsaufwand und die Erweiterung und Umbau der einzelnen Seiten ist schwer bis unm�glich.

Durch die Bereitstellung aller bisherigen (und zus�tzlicher) Funktionalit�ten soll Incave.de schneller, einheitlicher und leichter erweiterbar werden.

Ausserdem sollen aktuelle und zuk�nftige Technolgien fr�hzeitig umgsetzt werden um eine moderne und ansprechende Seite zu erstellen, auch wenn dies bedeutet, dass bestimmte Benutzergruppen ggf. nicht den vollen Umfang der Seite nutzen k�nnen. Dies sollte aber auf Grund der verwendeten Technolgien ann�hrend ausgeschlossen werden und nur in begr�ndeten Einzelf�llen auftreten.

Die gesamte Seite soll dem XHTML 1.0 und CSS2.0 Standard entsprechen. Eine Beschr�nkung auf bestimmte Browser soll ausgeschlossen werden und Workaround f�r unterschiedliche Browser vermieden werden.

\subsection{Zeit}
Grob geplant ist eine Umstellung bis zum 4.4.2007 - dem 5j�hrigen Geburstag von Incave. Dann muss aber auch der Bild-Mail-Versand funktionieren. Der Upload von Bildern f�r Galerien �ber ein Javaaplet ist nicht systemkritisch und kann auch sp�ter hinzugef�gt werden.

Auch die verschiedenen Designs werden erst nach und nach Einzug ins System halten.

\subsection{Verweise auf sonstige Ressourcen oder Quellen}
\begin{itemize}
 \item XHTML: http://de.selfhtml.org/html/xhtml/unterschiede.htm 
 \item CSS-Referenz: http://de.selfhtml.org/navigation/css.htm
 \item J/MySQL5: http://dev.mysql.com/downloads/connector/j/5.0.html
 \item Tomcat 5.5: http://tomcat.apache.org/tomcat-5.5-doc/index.html 
 \item DatenschutzG: http://www.gesetze-im-internet.de/bdsg\_1990
 \item Personenbezogene Daten: http://www.hss.de/datenschutz.shtml
 \item WBB-Forum: http://www.woltlab.de/products/burning\_board\_lite/index.php
 \item JFreeChar: http://www.jfree.org/jfreechart
\end{itemize}

